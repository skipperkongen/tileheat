% !TEX root = ./tileheat.tex
\begin{abstract}
Public geospatial services are now commonly available on the Web. These services often render
maps to users by dividing the maps into tiles. Given that geospatial services experience significant
user load, it is desirable to pre-compute tiles at a time of low load in order to increase overall
performance. Based on our analysis of the request log of a public geospatial service provider,
we observe that times of low load occur with a periodic pattern. 
%This pattern supports that pre-computed tiles are refreshed during low load. 
In addition, our analysis shows that tile access
patterns exhibit strong spatial skew.

Based on these observations, we propose an adaptive strategy restricting the set of tiles that
are pre-computed to fit the low load time window. Ideally, the restricted tile set should deliver
performance comparable to the full tile set. To achieve this result, tiles should be selected based on
their expected popularity. Our key observation is that the popularity of a tile can be estimated by
analyzing the tiles that users have previously requested. Our adaptive strategy constructs heatmaps
of previous requests and uses this information to decide which tiles to pre-compute. We examine
two alternative heuristics, one of which exploits that nearby tiles have a high likelihood of having
similar popularity. We evaluate our methods against a real production workload, and observe that
the latter heuristic achieves a 25\% increase in the hit ratio compared to current methods, 
without pre-computing a larger set of tiles.
\end{abstract}
